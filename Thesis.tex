\documentclass[12pt, twoside]{report}

	\usepackage[utf8]{inputenc}%кодировка
	\usepackage[russian]{babel}% разрешает делать подписи на русском
	\usepackage{amsmath}
	\usepackage{amsfonts}
	\usepackage{amssymb}
	\usepackage{graphicx}	
	\usepackage{wrapfig}
	\usepackage[margin=1in]{geometry}
	\usepackage{float}
	\usepackage{indentfirst} %красная строка первого абзаца в разделе(если хочешь писать без нее надо \noindent использовать)
	\usepackage{physics}
	\linespread{1.3}% создаем полуторный междустрочный интервал во всем тексте
% всё для того, чтобы ссылочки красивые были
	\usepackage[superscript,biblabel,ref]{cite}
	\usepackage[hidelinks, linktoc=all, backref=page, russian]{hyperref}
\renewcommand\backreftwosep{, } %добавка в списке литературы подписи с ссылкой на страницу
\renewcommand*{\backreflastsep}{, } 					%где ссылаешься на этот источник
\renewcommand*{\backref}[1]{}
\renewcommand*{\backrefalt}[4]{
	\ifcase #1 %
	\or (ссылка на странице #2)%
	\else (ссылки на страницах: #2)%
	\fi}



\begin{document}

\begin{titlepage}
	\begin{center}
Государственное образовательное учреждение высшего профессионального образования\\
Московский Физико-Технический Институт (Государственный Университет)\\
Факультет общей и прикладной физики\\
Кафедра физики и технологии наноструктур\\
	\end{center}

\vspace{5cm}
	\begin{center}
Выпускная квалификационная работа бакалавра\\~\\
\Large \textbf{Реализация двухкубитной схемы через перестраиваемый резонатор}
	\end{center}

\vspace{4cm}

	\begin{center}
		\begin{minipage}{0.45\textwidth}
			\begin{center}
Выполнил студент:\\~\\~\\
Цицилин И. А.\\
группа 425
			\end{center}
		\end{minipage}
		\begin{minipage}{0.45\textwidth}
			\begin{center}
Научный руководитель:\\~\\~\\
Беседин И.С.\\
??
			\end{center}
		\end{minipage}
	\end{center}

\vspace{\fill}


	\begin{center}
г. Москва, 2018
	\end{center}

\end{titlepage}

\newgeometry{right=1in, left = 1.2in, top=1in, bottom=1in}

\tableofcontents
	\newpage

\chapter{Введение}
а ваще тут надо написать что квкомп важно, про физ реализации\\
 Здесь и в дальнейшем мы будем говорить только о сверпроводящих элементах и о куче воды заточенной под методичку\\ и пока я сюда пишу то, что не знаю куда писать
 Хочется отметить, что практически все рассмотрения поведения(динамики) кубитов, ввиду того, что трансмон--слабоангармонический осциллятор, считают ( в первом приближении), что \textbf{кубит-- гармонический осциллятор}, поэтому по сути мы практически всегда рассматриваним взаимодействие гарм.осц.-- гарм.осц. К примеру такой эффект как антикроссинг (avoided crossing) можно наблюдать и в обычных классических системах \cite{Novotny2010}. Когда резонатор связывается с кубитом, то уже не существует по отдельности резонатора и кубита, существует только система кубит--резонатор. Их дисперсионные зависимости,существовавшие по отдельности, становятся одним целым (картиночки). Это явление также широко известно в экситонных поляритоннах и др. системах. По аналогии с экситонным поляритоном, где какое-то время существует экситонн, а другое поляритон, можно сказать что наше электромагнитное возбуждение(нафиг эту аналогию, лучше просто про связ.кол.эл.контуры) перетекает между резонатором и кубитом. Чем дольше возбуждение существует в этой системе, ( и тут я запутался, возбуждение гоняет только на резонансных частотах, а резонансная частота теперь то у них общая, и когда мы посылаем сигнал на частоте резонатора мы делаем возбуждение только в резонаторе, в кубите оно затухает, видимо это вносит дополнительную декогеренцию, т.к. в резонаторе начинают гулять два сигнала теперь и резонаторный чисто и кубитно-резонаторный)( ОГО ЧЕ ПРИДУМАЛ, НАДО СПРОСИТЬ У ИЛЬИ) -- тем дольше "живет" кубит, тем больше $T_1$
	\section{Эффект Джозефсона.SQUID}
	Джозефсоновская энергия имеет вид 
\begin{equation}
	E_J(\Phi)=-E_J\cos{\frac{\pi\Phi}{\Phi_0}}
	\label{eq:jos}	
\end{equation}	
	 где $E_J=\frac{\Phi_0I_c}{2\pi}$,$I_c$-- критический ток через джозефсоновский переход,$\Phi_0$-- квант магнитного потока. 
	В случае асимметричного СКВИДА
\begin{center}
	$E_J(\Phi)=\sqrt{((E_{J1}+E_{J2})\cos{\frac{\pi\Phi}{\Phi_0}})^2+
	((E_{J1}-E_{J2})\sin{\frac{\pi\Phi}{\Phi_0}})^2}$
\end{center}	 , $E_{Ji}=\frac{\Phi_0I_i}{2\pi}$,$I_i$-- критический ток через $i$-ый джозефсоновский переход.
Коэффициент асимметричности определяется как $d=\frac{E_{J1}-E_{J2}}{E_{J1}+E_{J2}}$. Такой тип СКВИДа хорошо подходит для кубитных измерений, так как в минимумах и максимумах частоты по полю  у трансмона($\frac{E_J}{E_C}\gg1$) наблюдается пониженая чувствительность к шумам. Т.е. используя асимметричный СКВИД мы увеличиваем число "рабочих точек"('sweet spots')\cite{Koch2007}.  
	\section{Сфера Блоха.Однокубитные гейты}

	\subsection{Кубит, связанный с резонатором}
	При рассмотрение всех эффектов в данном разделе мы считаем кубит-трансмон-слабоангармоничным оссцилятором, поэтому в первом приближении будем считать его гармоническим осциллятором. Рассмотрим эквивалентное представление кубита в виде электрической цепи.
\begin{center}
$L(\dot{\phi},\phi)\Rightarrow H(q,\phi)$\\
	$L=T-\Pi, T=\frac{1}{2}(C_{q}\dot{\phi}_1^2 + C_1(\dot{\phi}_2 - \dot{\phi}_1)^2 + Cr\dot{\phi}_2^2)$\\
	$\Pi= -E_J\cos{\frac{\pi\Phi}{\Phi_0}} \approx \frac{\phi^2}{2L_q}+const$ в кванте потока
\end{center}
	теперь преобразуем кинетическую энергию в переменные гамильтониана (преобразование Лежандра)
\begin{center}	
	$T=\frac{1}{2}\vec{\dot{\phi}}^TC\vec{\dot{\phi}} = \frac{1}{2}\vec{q^T}C^{-1}\vec{q}$
	$T=\frac{1}{2}\frac{C_1+C_r}{C_qC_r + C_1(C_q + C_r)}q_q^2+
	\frac{C_1}{C_qC_r + C_1(C_q+C_r)}q_qq_r+
	\frac{1}{2}\frac{C_1 + C_q} {C_qC_r + C_1(C_q + C_r)}q_r^2$\\
\end{center}
Полный вид гамильтониана кубит-резонатор имеет вид
\begin{center}	
$H=\frac{1}{2}\frac{C_1+C_r}{C_qC_r + C_1(C_q + C_r)}q_q^2+
	\frac{C_1}{C_qC_r + C_1(C_q+C_r)}q_qq_r+
	\frac{1}{2}\frac{C_1 + C_q} {C_qC_r + C_1(C_q + C_r)}*q_r^2 + \frac{\phi_q^2}{2L_q}+\frac{\phi_r^2}{2L_r}$
\end{center}
учитывания, что обычно $C_1 \ll C_q,C_r$,то разлагая до первого порядка  
\begin{center}	
$H=\frac{1}{2}(\frac{1}{C_q}-\frac{C_1}{C_q^2})q_q^2+\frac{1}{2}\frac{\phi_q^2}{L_q}+$
$\frac{1}{2}(\frac{1}{C_r}-\frac{C_1}{C_r^2})q_q^2+\frac{1}{2}\frac{\phi_r^2}{L_r}+$
$\frac{C_1}{C_qC_r}q_rq_q$,\\
где$\frac{C_1}{C_qC_r}q_rq_q$ - выражение определяющее связь кубит-резонатор\\
\end{center}
Проквантуем теперь наши осцилляторы согласно обычным правилам:
\begin{center}
$Z_i=\sqrt{\frac{L_i}{C_i}}$
$q_i=\sqrt{\frac{\hbar}{2Z_i}}(\hat{a^{\dagger}}+\hat{a})$
$\phi_i=i\sqrt{\frac{\hbar Z_i}{2}}(\hat{a^{\dagger}}-\hat{a})$ \\ 
\end{center}
для кубита надо сделать замену  $\hat{a^{\dagger}}\Rightarrow \hat{\sigma^-} $ и
 $\hat{a}\Rightarrow \hat{\sigma^+}$ 
Тогда гамильтониан примет вид:
\begin{equation}
\hat{H}=-\frac{\hbar\omega_q }{2}\sigma_z+\hbar\omega_r\hat{a}\hat{a^{\dagger}}+\hbar g 
(\hat{\sigma^-}+\hat{\sigma^+})(\hat{a^{\dagger}}+\hat{a})
\label{eq:Jc}
\end{equation}
где $\omega_q=\frac{1}{\sqrt{L_qC_q}}$,
$\omega_r=\frac{1}{\sqrt{L_rC_r}}$
$g=\frac{C_1}{2\sqrt{C_qC_r}}\sqrt{\omega_r\omega_q}$ в нулевом приближении по малости ёмкости связи.
Также заметим из того, что выражения для $q_q , \phi_q$ выражаются через $\sim \sigma_x ,\sim \sigma_y$ соответственно
,в случае ёмкостного связывания кубитов между собой имеем связывание их $\sigma_x$  матриц.

\section{Дисперсионное считывание}

Рассмотрим наш гамильтониан в вращающейся системе координат
\begin{center}

$\hat{H}=\omega_r\hat{a}\hat{a}^{\dagger}+\frac{1}{2}\omega_q\sigma_z+
g\sigma_x(\hat{a}+\hat{a}^{\dagger})$

\end{center}
для этого применим унитарное преобразование
\begin{center}
 $U=exp(i\omega_rt\hat{a}\hat{a}^{\dagger}+i\frac{\omega_q}{2}t\sigma_z)$
\end{center}
\begin{center}
$H_{tr}=UHU^{\dagger}+i\dot{U}U^{\dagger}$,
$i\dot{U}U^{\dagger}=i(i\omega_rt\hat{a}\hat{a}^{\dagger}+i\frac{\omega_q}{2}t\sigma_z)UU^{\dagger}=-\omega_rt\hat{a}\hat{a}^{\dagger}-\frac{\omega_q}{2}t\sigma_z$
\end{center}
первые 2 члена в нашем гамильтониане коммутируют с $U$(т.е. остаются такими же), значит нам надо понять как меняется при таком преобразовании наш последний член.
Воспользуемся леммой Хаусдорфа:
\begin{equation}
e^{\hat{A}} \hat{B} e^{-\hat{A}} = \hat{B}+[\hat{A},\hat{B}]+\frac{1}{2}[\hat{A},[\hat{A},\hat{B}]+...
\label{eq:Hausdorf}
\end{equation}
Отметим сразу, что примении этой леммы требует определённой осторожности, так как пренебрегать следующими коммутаторами стоит с умом(дальше увидим, что пренебрежние следующим по ходу коммутатором меняет значение в два раза).
$[\sigma_-,\sigma_z]=2\sigma_-$
\begin{center}
$U\sigma_-U^{\dagger}=\sigma_- +
i\frac{\omega_q}{2}t[\sigma_z,\sigma_-]+
\frac{1}{2!}(i\frac{\omega_q}{2}t)^2[\sigma_z,[\sigma_z,\sigma_-]]+...
=\sigma_-(1- i\frac{\omega_q}{2}t+\frac{1}{2!}(-i\frac{\omega_q}{2}t)^2+...)=\sigma_-e^{-i\omega_qt}$
\end{center}
\newpage
\begin{center}
$U\sigma_+U^{\dagger}=(U\sigma_-U^{\dagger})^{\dagger}=
\sigma_+e^{i\omega_qt}$
\end{center} 
так же, т.к. $[a,a^{\dagger}a]=a$
\begin{center}
$UaU^{\dagger}=ae^{-i\omega_rt},
Ua^{\dagger}U^{\dagger}=a^{\dagger} e^{i\omega_rt}$\\
$H_{tr}=g(\sigma_-e^{-i\omega_qt}+\sigma_+e^{i\omega_qt})
(ae^{-i\omega_rt}+a^{\dagger}e^{i\omega_rt}) =
g(a\sigma_+e^{i(\omega_q-i\omega_r)t}+
a^{\dagger}\sigma_-e^{i(\omega_r-\omega_q)t}+
a\sigma_-e^{-i(\omega_q+\omega_r)t}+
a^{\dagger}\sigma_+e^{i(\omega_q+\omega_r)t})$
\end{center}
,если частоты удовлетворяют соотношению $|\omega_q-\omega_r| \ll \omega_q + \omega_r$ и $g \ll \omega_q,\omega_r$
, то можно отбросить быстроосциллирующие члены - это так называемое RWA, и получить гамильтониан вида:
\begin{equation}
\hat{H}=\omega_r\hat{a}\hat{a}^{\dagger}+\frac{1}{2}\omega_q\sigma_z+
g(\hat{a}\sigma_-+\hat{a}^{\dagger}\sigma_+)
\label{eq:RWA}
\end{equation}
Покажем почему так можно сделать.\\
Производная по времени оператора имеет вид в представлении
Гейзенберга, если гамильтониан системы имеет вид \ref{eq:Jc}
\begin{center}
$
\dot{a}=\frac{i}{\hbar}[H,a]
$
\end{center}
Используя $[a,a^{\dagger}]=1,[a,a]=0$ \\
$\dot{a}(t)=i\omega_r a(t) +ig(-\sigma_ -(t) e^{i(\omega_r-\omega_q)t} - \sigma_+(t)e^{i(\omega_q+\omega_r)t})$
решим это уравнение методом последовательных приближений. Считая, что в первом приближении осцилляторы не взаимодействуют,решим уравнение с $g=0$ и проинтегрируем его
\begin{center}
$a(t)=ae^{i\omega_rt}+g(\frac{\sigma_-e^{i(\omega_r-\omega_q)t}}{\omega_r-\omega_q}+\frac{\sigma_+e^{i(\omega_r+\omega_q)t}}{\omega_r+\omega_q})$
\end{center}
и если $|\omega_q-\omega_r| \ll |\omega_r+\omega_q|$ и
$g \ll |\omega_r+\omega_q|$, то в первом приближении
эволюция оператора  определяется как 
$a(t)=ae^{i\omega_rt}+g(\frac{\sigma_-e^{i(\omega_r-\omega_q)t}}{\omega_r-\omega_q})$, аналогично для $\sigma_-^+(t)$.
Следовательно гамильтониан, определяющий такую же эволюцию операторов, имеет вид (\ref{eq:RWA}).

Теперь рассмотрим как происходит считывание состояния кубита - дисперсионное считывание. Рассмотрим гамильтониан \ref{eq:RWA}, в случае дисперсионного считывания $g\ll |\Delta|=|\omega_q-\omega_r|$,т.е. когда у нас есть большая отстройка по частоте между кубитом
и резонатором разделим его на 2 части, отвечающие за взаимодействие и остальное:
\begin{center}
$H=H_0+H_1,
H_0=\omega_r\hat{a}\hat{a}^{\dagger}+\frac{1}{2}\omega_q\sigma_z,
H_1=g(\hat{a}\sigma_-+\hat{a}^{\dagger}\sigma_+)
$
\end{center} 
Выберем унитарное преобразование таким, чтобы $U=exp(S),
[S,H_0]=-H_1$, тогда по лемме Хаусдорфа \ref{eq:Hausdorf}
\begin{center}
$e^SHe^(-S) = H+ [S,H]+\frac{1}{2}[S,[S,H]]+...=
H_0+H_1+[S,H_1]-H1+\frac{1}{2}[S,-H_1]+...=
H_0+\frac{1}{2}[S,H_1]+...$
\end{center}
Покажем, что если взять $S=\frac{g}{\Delta}(a\sigma_+-a^{\dagger}\sigma_-)$, то этого можно достичь.
Учитывая следующие соотношения
\begin{center}
$
[a\sigma_+-a^{\dagger}\sigma_-,a^{\dagger}a]=
a\sigma_++a^{\dagger}\sigma_-$\\
$
[a\sigma_+-a^{\dagger}\sigma_-,\sigma_z]=
-2(a\sigma_++a^{\dagger}\sigma_-)$\\
$
[a\sigma_+-a^{\dagger}\sigma_-,a\sigma_++
a^{\dagger}\sigma_-]=2N\sigma_z
$
$
N=a^{\dagger}a+\frac{1}{2}+\frac{\sigma_z}{2}
$ 
\end{center}
с учётом того, что $g \ll |\Delta|$ обозначим  $q=\frac{g}{\Delta}$ 
\begin{center}
$
UaU^{\dagger} = a +q[a\sigma_+-a^{\dagger}\sigma_-,a]+
\frac{q^2}{2}[a\sigma_+-a^{\dagger}\sigma_-,[a\sigma_+-a^{\dagger}\sigma_-,a]]+O(q^3)=
a+q\sigma_-+\frac{q^2}{2}a\sigma_z+O(q^3)
$

$
U\sigma_zU^{\dagger} = \sigma_z +q[a\sigma_+-a^{\dagger}\sigma_-,\sigma_z]+
\frac{q^2}{2}[a\sigma_+-a^{\dagger}\sigma_-,[a\sigma_+-a^{\dagger}\sigma_-,\sigma_z]]+O(q^3)=
\sigma_z-2q(a\sigma_++a^{\dagger}\sigma_-)-q^2
[a\sigma_+- 
a^{\dagger}\sigma_-,a\sigma_++a^{\dagger}\sigma_-]+O(q^3)=
\sigma_z-2q(a\sigma_++a^{\dagger}\sigma_-)-q^2\sigma_z
(1+2a^{\dagger}a)+O(q^3)
$
где использовалось, что $\sigma_+\sigma_-=
\frac{1+\sigma_z}{2}$
$U\sigma_-U^{\dagger}=\sigma_-+qa\sigma_z+O(q^2)$
$
UHU^{\dagger}=\omega_r a^{\dagger}a+
q\omega_r(a\sigma_++a^{\dagger}\sigma_-)+
q^2\omega_r\sigma_z(a^{\dagger}a+\frac{1}){2})+\frac{\omega_q}{2}\sigma_z-
q\omega_q(a\sigma_++a^{\dagger}\sigma_-)
-q^2\omega_r\sigma_z(a^{\dagger}a+\frac{1}{2})+
g(a\sigma_+a^{\dagger}\sigma_-)+qg\sigma_z(2a^{\dagger}a+1)+
gO(q^2)=\omega_r a^{\dagger}a+\frac{\omega_z}{2}\sigma_z+
qg\sigma_z(a^{\dagger}a+\frac{1}{2})+O(q^2)
$
В результате получаем, что наш гамильтониан диагонализуется
в первом приближении по $\frac{g}{\Delta}$
\begin{equation}
H=(\omega_r+\frac{g^2}{\Delta}\sigma_z)a^{\dagger}a + 
\frac{1}{2}(\omega_q+\frac{g^2}{\Delta})\sigma_z
\end{equation} 
Т.е. мы имеем 2 невзаимодействующие системы, но при этом мы можем измерять состояние кубита $\sigma_z$ через измерение изменения
частоты резонатора $\omega_r+\frac{g^2}{\Delta}\sigma_z$, поэтому нам надо соблюдать несколько условий : $g\ll|\Delta|$ и 
$\frac{g^2}{\Delta}$ измеримая величина ( обычно $g\approx10$МГц,$|\Delta|\approx1$ГГц, значит изменение частоты $\approx0.1$МГц)
\end{center}
		\subsection{Два кубита, связанные через один резонатором}
ссылкa\cite{Gu2017}





\chapter{Теоретические сведения} \label{chap:theory}
	\section{Резонатор}
(вставлю картинку связанного кубита-резонатора)
связь кубит-резонатор в представлении сосредоточенных электрических элементов
	\begin{equation}
	g_{qr}=\frac{C_c}{2\sqrt{C_q C_r}}\sqrt{\omega_r \omega_q}
	\label{eq:g_qr}
	\end{equation}
выражение для константы связи, полученное через квантовомеханический рассчет, вы можете найти в \cite{Koch2007}(надо бы его сюда тоже вписать и показать, что вычисление через классику и через кв.мех должны совпадать, чисто для себя проверить)\\
Резонатор-гармонический осциллятор, он используется как фильтр, он пропускает только в определенной области частот близкой к резонансу,т.е. он выполняет не только функцию пропускания сигнала, но и блокирует пропускание 
различных шумов\cite{Naturwissenschaften2013}. Также он запасает в себе энергию электромагнитного поля и тем самым помогает её транспортировать между кубитами, связанными через один резонатор \cite{Kelly2015}
(можно здесь написать про внутреннюю и внешнюю добротность , т.к. я же каппу через прогу Ильи считаю, хорошо бы объяснить что это) \\
Резонансные частоты CPW определяются по формулам:
	\begin{align}
	\label{resfr}
	f^{(0)}_r=\frac{c}{\sqrt{\varepsilon_{eff}}}\frac{2p-1}{4l}, 
	f^{(0)}_r=\frac{c}{\sqrt{\varepsilon_{eff}}}\frac{2p}{4l}
	\end{align}
,где $v_{ph}=\frac{c}{\sqrt{\varepsilon_{eff}}}$-фазовая скорость распространения волны в резонаторе. В \eqref{resfr} первая описывает резонатор c закороченным одним концом $(\lambda/4)$ и с открытым другим, вторая описывает резонатор с обоими закороченными концами или с обоими открытыми концами $(\lambda/2)$. 
	
ДОБАВИТЬ ДОБРОТНОСТЬ\\
Резонатор представляет из себя копланарный волновод(надо поменять их местами по секциям) определенной длины, он является распределенным элементом, имеет несколько резонансных частот и при частоте близкой к резонансной может быть представлен как набор следующих сосредоточенных элементов:
	\begin{equation}
	C_r=\frac{C_0 l}{2},
	L_r=\frac{8L_0 l}{\pi^2},
	R_r=\frac{Z_{CPW}}{\alpha_0 l}
	\end{equation}
где $Z_{CPW}=\sqrt{\frac{L_0}{C_0}}$-характеристический импеданс компланарного волнового резонатора, внутренние потери в резонаторе(квазичастицы, излучение и т.п.) представляются в виде $R_r$, которое имеет прямое отношение к внутренней добротности резонатора $Q^i$, за связывание с передающей линией с импедансом $Z_0$ отвечает внешняя добротность $Q^e$.

(это я описываю, чтобы потом описать затухание в линию)
 При частотах близких к резонансным мы будем всегда работать. Используя преобразование Нортона\cite{Goppl2008} мы можем переделать нашу эквивалентую схему в схему с параллельным подключением
при этом  
	\begin{align}
	C^*&=\frac{C_c}{1+\omega_0^2 C_c^2 Z_0^2} \approx C_c
	Z^*&=\frac{1+\omega_0^2 C_c^2 Z_0^2}{\omega_0^2 C_c^2 Z_0^2} \approx \frac{1}{\omega_0^2 C_c^2 Z_0^2} 
	\end{align}
т.к. $Z_0\sim$, то$\omega_0 C_c Z_0 \ll 1$
	\subsection{Эффект Парселла}
про парселла\cite{Koch2007}	
	\subsection{Связывание кубита с резонатором}
про клешню\cite{Sank2014}\\
Связывание кубита с резонатором -- ёмкостное,	

\newpage
	\section{Копланарный волновод}%тут вроде всё
Копланарный волновод представляет из себя тонкую полосу металла -- 'сигнал' -- шириной W, отделенную от земли щелью G и длинной l.
	\begin{figure}[h]
		\begin{center}
\includegraphics[width=0.5\textwidth]{plots/cpw}
\caption{Копланарный волновод}
		\end{center}
	\end{figure}
Если d $\ll$ h,где h- высота подложки, т.е. если слой сверхпроводника очень тонкий,можно приблизительно считать,что\cite{Goppl2008}
	
	%&-значек, по которому происходит выравнивание
	\begin{align*}
	C'&=4\varepsilon_0\varepsilon_{eff}\frac{K(k_0)}{K(k'_0)},
	L'=\frac{\mu_0}{4}\frac{K(k'_0)}{K(k_0)}\\
	Z_0&=\sqrt{\frac{L'}{C'}}	=\frac{\mu_o c}{4\sqrt{\varepsilon_{eff}}}\frac{K(k'_0)}{K(k_0)}
	\end{align*}
где $k_0=\frac{W}{W+2G},k'_0=\sqrt{1-k_0^2}$, K-полный нормальный эллиптический интеграл Лежандра 1-ого рода, который считается численно.
Т.е. импеданс передающей линии зависит от двух параметров от эффективной диэлектрической проницаемости $\varepsilon_{eff}=\frac{\varepsilon_{\text{среда сверху}}+\varepsilon_{\text{среда снизу}}}{2}$, где $\varepsilon_{\text{среда сверху}}=\varepsilon_{\text{вакуума}}=1$,а в качестве подложки используют кремний или сапфир $\varepsilon_{\text{кремний}}\approx11.45$(в гопл написано 11.6 и что интересно у них эпсилон эффективная не 6.225, а 5.22 с погрешностью в 3процента???).Посчитать импеданс по этим формулам, используя удобный интерфейс вы можете на этом сайте(ссылка на сайт ильи), где имеется описание. В сверхпроводнике помимо геометрической(магнитной) индуктивности существует кинетическая индуетивность\cite{Goppl2008}, которая при некоторых условия может быть порядка геометрической. Рассмотрим характерное значение $L^k$ для Al плёнок,которые используются в наших схемах.(ссылки в гоппл надо посмотреть когда интернет будет) Но значение кин.индуктивности на 2 порядка меньше чем геом.индуктивности, поэтому ею можно пренебречь.
(ЗАБОТАЙ КИН.ИНДУКТИВНОСТЬ СОЗДАЙТЕ ЕЁ УЖЕ НАКОНЕЦ)
\newpage
	\section{Линия передачи}%выведу здесь для S11 зависимость
	




\chapter{Экспериментальная часть} \label{chap:exp}
Для разработки дизайна надо понять, что именно и как именно мы хотим исследовать на нём. Наша задача-
связать два кубита. Во многих работах связывание происходит через резонатор\cite{Kelly2015}(и еще добавлю). 
В нашей разработке хотелось бы реализовать в дальнейшем двухкубитный гейт -- CNOT, для реализации универсального квантогово компьютера. Чтобы варьировать связь между кубитами можно использовать нелинейный резонатор -- ещё один вспомогательный кубит(ссылка на стр в теории где показываю, что кубит нлр), меняя его частоту при помощи Z-гейта(ссылка на стр в теории или дальше, где я могу поговорить о том, как z гейт делать вообще), мы можешь варьировать связь, т.к. связь кубит--резонатор имеет вид \eqref{eq:g_qr},т.е. $g_{qr} \sim \sqrt{\omega_r}$, а значит использовать кубит как нелинейный связывающий резонатор целесообразно.Также исследование трехкубитной схемы необходимо для дальнейшей масштабируемости цепочки кубитов.
т.е. мы имеем первый набросок дизайна
(вставка графика)
(где-то я должен в теориии рассказать что такое иксмон и зачем нам его удобно использовать)
	
	\section{Разработка дизайна образца}
	Определим теперь какие нам необходимы допольниительные элементы для
	\begin{enumerate}
	\item считывания состояний кубитов
	\item возбуждения кубитов
	\item взаимодествия кубит--кубит
	\item взаимодействия кубит-- резонатор
	\end{enumerate}
	
		\subsection{Считывание}
Для считывания будем измерять отражение сигнала в линии передачи (\textit{transmission line}) -- $ S_{11}$
надо сюда вставить как зависит от расстояния до конца линии напряжение на ней или что вообще 
можно сказать про считывание)\cite{GlebMaster}(можно вставить умную зависимость для S11!)
Надо уметь считывать состояния кубитов по одному, для этого используются резонаторы на разных частотах \ref{chap:theory}(ссылка говно!) для каждого кубита,т.к. мы используем дисперсионное считывание(ссылочка на теорию где я говорю, что это такое), то наш резонатор отстроен по частоте от кубита, в результате, мы получаем, что сигнал, приходящийся на кубит сильно ослаблен, если его возбуждать через резонатор, и надо подавать больше мощности, если резонатор высокодобротный, то сигнал может быть слишком слабый.
	\begin{figure}[h]
		\begin{center}
\includegraphics[width=0.5\textwidth]{plots/FR}
\caption{АЧХ кубита и резонатора}
		\end{center}
	\end{figure}
Так же, если у нас есть два кубита на близкой частоте, то подавая сильный сигнал через линию передачи для возбуждения одного кубита, мы можем внести дополнительные нежелательные возбуждения в другой кубит.
		
		\subsection{Возбуждение}
Чтобы делать гейты(возбуждать кубиты) надо подавать сигнал на кубит на его резонансной частоте. Есть два способа: подавать сигнал в линию передачи,но,чтобы сигнал прошел на кубит,ему надо пройти через резонатор, или 
можно сделать отдельную линию передачи -- 'драйв', которая будет сильнее всего связана с только одним кубитом. Мы используем второй вариант -- драйв.Драйвы должны быть разнесены таким образом, чтобы можно было возбуждать один кубит разумной мощностью и при этом не возбуждать соседние кубиты( минимальный crosstalk). Наличие драйва -- наличие дополнительного источника релаксации состояния кубита, т.е. он должен иметь малую ёмкостную связь с электродом кубита.
(тут или не тут вставлять картинки из лэйаут и говорить как я там все моделировал и считал или пока вода?)
 		
 		\subsection{Взаимодействие кубит--центральный кубит}
Связь кубит--кубит -- $g_{qq}$ -- определяет время, за которое может переходить возбуждение с одного кубита на другой, т.е. определяет время двухкубитного гейта\cite{Rigetti2010} $t_{gate}\sim \frac{1}{g_{qq}}$. Использовать значение этой связи в данной работе мы будем как связь кубит--центральный кубит(наш нелинейный резонатор) 
\newpage	
		\subsection{Взаимодействие кубит--резонатор.Однотоновая спектроскопия}
Чтобы считывать состояние кубита через резонатор, он должен быть связан с константой связи порядка 20--30 Мгц,чтобы частота резонатора изменялась на разумную величину. Для измерения частот резонатора нам необходим векторный анализатор цепей. Подаем сигнал и смотрим как меняется его прохождение ($S_{21}$).Существует 3 вида зависимости резонансной частоты от внешнего потока, которые мы можем наблюдать:
\begin{enumerate}
\item Зависимоть резонатора от внешнего поля - прямая. Это может быть в случае если 
	\begin{enumerate}
	\item Что-то пошло не так и не получились джозефсоны -- $g=0$, т.е у нас нет связанной системы кубит-резонатор.

	\item Кубит находится слишком низко(высоко) по частоте относительно кубита, на столько, что мы не можем заметить изменение частоты резонатора при изменении состояния кубита.

	\item Тестовый резонатор - g=0, т.к. кубита там просто нет.
	\end{enumerate} 
	\begin{figure}[H]
		\begin{center}
		\includegraphics[width=0.5\textwidth]{plots/test_resonator}
		\caption{Прямая зависимость частоты резонатора от потока}
		\end{center}
		\label{fig:test_res}
	\end{figure}
\item Если частота кубита лежит ниже частоты резонатора, то зависимость резонатора от внешнего поля - прямая с подъёмами в точках максимума частоты кубита от внешнего потока.
	\begin{figure}[H]
		\begin{center}
		\includegraphics[width=0.5\textwidth]{plots/res_qub}
		\caption{Зависимость частоты резонатора от потока в случае $g\neq0$ и $\omega_r>\omega_q$. Синей кривой указана верхняя часть дисперсионной зависимости кубита}
		\end{center}
	\end{figure}
Аналогично, только теперь впадина, если частота кубита выше частоты резонатора.
\item  Если частота кубита пересекается с частотой резонатора.
	\begin{figure}[H]
		\begin{center}
		\includegraphics[width=0.5\textwidth]{plots/anti_sim}
		\caption{Зависимость частоты резонатора от потока в случае $g\neq0$ и пересечения частот кубита и резонатора.(avoided crossing)}
		\end{center}
	\end{figure}
	В таком случае можно найти связь между кубитом и резонатором если найти минимальное расстояние по вертикали между двумя максимумами пропускания резонатора в окрестности точки пересечения.
	\begin{figure}[H]
		\begin{center}
		\includegraphics[width=0.5\textwidth]{plots/anti_sim_slice}
		\caption{Нахождение величины связи кубит-резонатор.\textit{ Картинка получена как срез при фиксированном потоке 1а.е. из предыдущей картинки}}
		\end{center}
	\end{figure}
\end{enumerate}
Описание взаимодействия двух связанных систем эффективно имеет одинаковый вид независимо от природы этих систем\cite{Novotny2010}. Однотоновая спектроскопия - сигнал $S_{21}$, описывается формулой 
\begin{equation}
S_{21}(\omega)=(a+\frac{\eta}{\kappa+i(\omega_r-\omega)+\frac{g^2}
{\gamma+i(\omega_q-\omega)}})e^{it}
\end{equation}
,где $e^{it},a$- набег фазы по проводам и неидеальность мира,$\eta$- связь передающей линии с резонатором,$\kappa$- обратное время жизни фотона в резонаторе($\frac{\omega_r}{Q}$, $\gamma$-обратное время жизни фотона в кубите($\frac{1}{T_1}$),$g$-- сила связи кубит-резонатор,$\omega$-частота подаваемого сигнала, $\omega_r$- частота резонатора, которая у него в случае $g=0$,$\omega_q$- частота кубита, которая зависит от внешнего потока(\ref{eq:jos}).  




\newpage
На рисунке \ref{fig:anticrossings} можно видеть антикроссинги кубитов с резонаторами.

\begin{figure}[H]
		\begin{center}
\includegraphics[width=0.9\textwidth]{plots/phase_crossing}
\caption{Антикроссинг резонаторов с кубитами}
		\end{center}
		\label{fig:anticrossings}
	\end{figure}
	
тут вы можете видеть приближенную версию, величина расщепления составляет порядка 2g.
Метод заключается в подаче сигнала на частоте резонатора и измерение его прохождения
\begin{figure}[H]
		\begin{center}
\includegraphics[width=0.9\textwidth]{plots/phase_crossing_1_qubit.png}
\caption{Антикроссинг резонатора с кубитом}
		\end{center}
	\end{figure}
	\label{fig:anticrossings}

	
\newpage
		\subsection{Двухтоновая спектроскопия}
При помощи двухтоновой спектроскопии можно найти зависимость частоты переходов кубита от внешнего потока - дисперсионная зависимость кубита. Метод основывается на дисперсионном считывании. Подаём два сигнала, один на частоте первого тона --частота резонатора при данном потоке через кубит (как можно заметить из предыдующей части частота резонатора эффективно зависит от внешнего потока, в случае наличия ненулевой связи кубит-резонатор), другой на частоте второго тона-- предположительные частоты переходов кубита. Когда кубит возбуждается, то т.к. мы используем дисперсионное считывание, частота резонатора будет меняться, а частота первого тона остается той же самой - меняется пропускание сигнала, что мы и детектируем. В случае кубита - трансмона, надо находит экстремумы дисперсионной зависимости, т.к.(МОЖНО ВСТАВИТЬ)
		
\newpage	
		
\subsection{Осцилляции Раби}
В случае взаимодействия классического электромагнитного излучения с двухуровневым атомом, возбуждение атома осциллирует со временем. Рассмотрим этот процесс в случае отсутствия затухания.

\begin{figure}[H]
		\begin{center}
		\includegraphics[width=0.3\textwidth]			  {plots/Rabi.png}
		\caption{Двухуровневая система с энергией 
		$\omega_0$, на которую подается монохроматическая волна с частотой $\omega$,с отстройкой по частоте $\Delta =
		\omega - \omega_0$,Раби частота $\Omega$}
		\end{center}
\end{figure}
Эволюция системы имеет вид:
\begin{center}
$i\hbar\dfrac{\partial {\Psi(r,t)}}{\partial{t}}=\hat{H}(r,t)$
\end{center}
волновая функция в любой момент времени может быть записана как:
\begin{center}
$\ket{\Psi(r,t)}=C_g(t)\ket{g}+C_e(t)\ket{e}e^{-i\omega_0 t}$
\end{center}
Гамильтониан $H=H_0+V$, где $V_0=dE$ -- возмущение системы падающей волной. Распишем ур-ие Шредингера:

\begin{center}
$i\hbar\dfrac{dC_g(t)}{dt}=C_e \bra{g}dE\ket{e}e^{-i\omega_0t}$\\
$i\hbar\dfrac{dC_e(t)}{dt}=C_g \bra{e}dE\ket{g}e^{i\omega_0t}$\\
\end{center}
Для случая плоской волны $E=\epsilon E_0 \cos{(kr-\omega t)}$
можно ввести параметр взаимодействия -- частота Раби: 
\begin{center}
$\Omega=\dfrac{E_0}{\hbar}\bra{e}d\epsilon \ket{g}$
\end{center}

\begin{center}
$i\hbar\dfrac{dC_g(t)}{dt}=C_e \hbar \Omega^*(\dfrac
{e^{i(\omega-\omega_0)t}+e^{-i(\omega+\omega_0)t}}
{2})$\\
$i\hbar\dfrac{dC_e(t)}{dt}=C_g \hbar \Omega(\dfrac
{e^{i(\omega+\omega_0)t}+e^{-i(\omega-\omega_0)t}}
{2})$
\end{center}
Используя RWA в случае если $|\omega-\omega_0|\ll \omega+\omega_0$, то система уравнений принимает вид:

\begin{center}
$i\hbar\dfrac{dC_g(t)}{dt}=C_e \hbar \Omega^*(\dfrac
{e^{i \Delta t}}
{2})$\\
$i\hbar\dfrac{dC_e(t)}{dt}=C_g \hbar \Omega(\dfrac
{e^{-i\Delta t}}
{2})$
\end{center}

\begin{center}
$\dfrac{d^2C_g(t)}{dt}-i\Delta\dfrac{dC_g(t)}{dt}+\dfrac{\Omega^2}{4}C_g(t)=0$\\
$\dfrac{d^2C_e(t)}{dt}+i\Delta\dfrac{dC_e(t)}{dt}+\dfrac{\Omega^2}{4}C_e(t)=0$
\end{center}
Решением этой системы диф.уравнений с постоянными коэффициентами будет в случае если в t=0 атом находился в основном состоянии $C_g(0)=1,C_e(0)=0$, то 
\begin{center}
$|C_e(t)|=\dfrac{\Omega}{\Omega^{'}}\sin{\dfrac{\Omega^{'}t}{2}}$, где $\Omega^{'}=\sqrt{\Omega^2+\Delta^2}$ 

\end{center}

Значит,вероятность перехода атома из основного состояния в возбужденное меняется по закону 
\begin{equation}
P_{1-2}=\frac{\omega_R^2}{(\omega-\omega_0)^2+\omega_R^2}
\sin^2({\frac{1}{2}\sqrt{(\omega-\omega_0)^2+\omega_R^2}})
\end{equation}
где $\omega_R$ -- частота Раби,$\omega_q$ -- частота кубита,$\omega$ -- частота нашего электромагнитого поля,падающего на двухуровневую систему. Правда, это формула не учитывает затухание кубита, то есть она ещё должна быть умножена на $e^{- \Gamma t}$.
Исследование осцилляций Раби нам необходимо, чтобы понимать какова величина $\pi$--импульса,потому что он же переворот спина , он же $X_{\pi}$, он же NOT гейт. \\
Можно еще добавить что раби частота линейно зависит от мощности....
\newpage
\subsection{Измерение времен релаксации Т1 и Т2}
Матрица плотности кубита имеет вид:
\begin{center}
$\rho = $
$\begin{pmatrix} \rho_{11}&\rho_{12}\\  \rho_{21}& \rho_{22}\end{pmatrix}$
\end{center}
В случае наличия затухания диагональные элементы содержат в себе $e^{-t/T1}$, а не диагональные $e^{-t/T2}$, эти 2 характерных временных масштаба определяют свойства кубита.
Методика их измерения представляет из себя такой же набор действий как и измерение одноименных параметров в ЯМР спектроскопии. Т.е. для того, чтобы найти $T_1$ надо сделать следующую последовательность импульсов...\\
А для $T_2 $ следующую.


Покажем, за что отвечает каждый параметр, для этого опять же рассмотрим взаимодействие двухуровневой системы с внешним классическим электромагнитным полем
\begin{center}
$H=H_0-dE, \hbar\omega_0=E_e-E_g$,
$H_0=\dfrac{\hbar\omega_0}{2}\begin{pmatrix} 1&0\\  0& -1\end{pmatrix}$\\
можно показать что оператор дипольного момента в базисе  функций $H_0$ имеет вид $d=\begin{pmatrix} 0&d\\  d^*& 0\end{pmatrix}$ следовательно весь гамильтониан 
$H=\begin{pmatrix} \dfrac{\hbar\omega_0}{2}&-(dE)\\  -(d^*E)& -\dfrac{\hbar\omega_0}{2}\end{pmatrix}$
\end{center}
как раз недиагональный вид оператора диполного момента делает возможным передачу энергии от атома полю\\
Матрицу плотности можно переписать в виде:
\begin{center}
$\rho = \dfrac{1}{2}+\begin{pmatrix} N/2&\rho\\  \rho^*& -N/2 \end{pmatrix}$
\end{center}
Смысл недиагональных элементов матрицы плотности вытекает если посчитать $\langle d \rangle = tr(d\rho)=2Re(d^*\rho)$, т.е. взаимодействие атома с электромагнитым полем определяется недиагональными элементами матрицы плотности, а значит и взаимодействие кубита с кубитом ( диполь-дипольное) также определяется ими. Следовательно, мы можем выполнять двухкубитные гейты только за время пока $\rho \neq 0$ , т.е. за время $T_2$

		


\newpage

	\section{Экспериментальная установка}
		\subsection{Криостат}
		\subsection{СВЧ-оборудование}
не буду про фабрикацию ниче говорить, нафиг




\newpage
	\section{Заказанные параметры двухкубитной схемы}
\noindent Считаем $\epsilon_{eff}=6.225$\\
\noindent 3 одинаковых перестраиваемых кубита\\
\noindent Из fastcup получаем слудющие значения для ёмкостей:
	\begin{center}
		\begin{tabular}{ | c | c | c | }
   		 	\hline
   			$C_Q,fF$ & $ C_{Q-R},fF$ &$ C_{Q-Q},fF$ \\ \hline
   		 	101 & 1.94 & 1.76 \\ \hline
		\end{tabular}
	\end{center}
\noindent Перестройка по частоте от $f_{01}^{min}$=4ГГц до $f_{01}^{max}$=6.5ГГц\\
$E_{C}=\frac{e^2}{2C_Q}$ = 191 МГц\\
$E_{J_{\Sigma}}^{min} = \frac{(hf_{01}^{min}+E_{C})^2}{8E_{C}}=$11.5ГГц \\
$E_{J_{\Sigma}}^{max} = \frac{(hf_{01}^{max}+E_{C})^2}{8E_{C}}=$29.3ГГц \\
$E_{J}=\frac{\Phi_0 I_c}{2\pi}\Rightarrow$ \\
$I_{c_{\Sigma}}^{min}=$23.1нА $I_{c_{\Sigma}}^{max}=$59нА$\Rightarrow$
получаем крит.токи через джозефсоны(полусумма и полуразность максимального и минимального)\\
$I_{c_{1}}=$41нА $I_{c_{2}}=$18нА\\
при площадях $S_1 = 0.235*0.117 \text{мкм}^2$  $S_2 = 0.156*0.078\text{мкм}^2$  \\
$j_2=\frac{I_{c_{2}}}{S_2}=1.49\frac{\mu A}{{\mu m}^2}$
	\begin{center}
		\begin{tabular}{ | c | c | c |}
   		 	\hline
   			  &1 &2 \\ \hline
    			 $I_c,nA$ & 41 & 18 \\ \hline
    	 		$S,{\text{мкм}}^2$ & 0.235*0.117 & 0.156*0.078  \\ \hline
    	 		 $j,\frac{\mu A}{{\mu m}^2}$  &\multicolumn{2}{|c|}{1.49}\\ \hline
		\end{tabular}
	\end{center}
	
Получились частоты перестройки с такими параметрами 
от $f_{01}^{min}$=4.17 ГГц у боковых и 4.13 ГГц у центрального до $f_{01}^{max}$=6.65ГГц у боковых и 6.6ГГц у центрального.

Диапазон констант связи кубит--центральный кубит в зависимости от частоты кубитов:
	\begin{center}
		\begin{tabular}{ | c | c | c |}
  	  	\hline
   		  &для $f_{01}^{min}$ &для $f_{01}^{max}$ \\ \hline
    		  $g_{Q-Q}$,МГц &35,2  & 56,2 \\ \hline
		\end{tabular}
	\end{center} 
	
Диапазон констант связи кубит--резонатор в зависимости от частоты кубита:
	\begin{center}
		\begin{tabular}{ | c | c | c |}
  	  	\hline
   		  f резонатора,ГГЦ&для $f_{01}^{min}$ &для $f_{01}^{max}$ \\ \hline
    		  6.85 &23,1 МГц & 29,2 МГц \\ \hline
    		  7.15 &24,1 МГц & 30,5 МГц \\ \hline
   	 	  7 &23,6 МГц & 29,9 МГц \\ \hline
		\end{tabular}
	\end{center} 
	
	\section{Полученные параметры двухкубитной схемы}
	\bibliographystyle{ugost2008ls}
	\bibliography{Thesis_Ivan.bib}
\end{document}
