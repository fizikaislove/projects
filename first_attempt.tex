\documentclass[12pt, twoside]{report}

	\usepackage[utf8]{inputenc}%кодировка
	\usepackage[russian]{babel}% разрешает делать подписи на русском
	\usepackage{amsmath}
	\usepackage{amsfonts}
	\usepackage{amssymb}
	\usepackage{graphicx}	
	\usepackage{wrapfig}
	\usepackage[margin=1in]{geometry}
	
	\usepackage{indentfirst} %красная строка первого абзаца в разделе(если хочешь писать без нее надо \noindent использовать)

	\linespread{1.3}% создаем полуторный междустрочный интервал во всем тексте
% всё для того, чтобы ссылочки красивые были
	\usepackage[superscript,biblabel,ref]{cite}
	\usepackage[hidelinks, linktoc=all, backref=page, russian]{hyperref}
\renewcommand\backreftwosep{, } %добавка в списке литературы подписи с ссылкой на страницу
\renewcommand*{\backreflastsep}{, } 					%где ссылаешься на этот источник
\renewcommand*{\backref}[1]{}
\renewcommand*{\backrefalt}[4]{
	\ifcase #1 %
	\or (ссылка на странице #2)%
	\else (ссылки на страницах: #2)%
	\fi}



\begin{document}

\begin{titlepage}
	\begin{center}
Государственное образовательное учреждение высшего профессионального образования\\
Московский Физико-Технический Институт (Государственный Университет)\\
Факультет общей и прикладной физики\\
Кафедра физики и технологии наноструктур\\
	\end{center}

\vspace{5cm}
	\begin{center}
Выпускная квалификационная работа бакалавра\\~\\
\Large \textbf{Реализация двухкубитной схемы через перестраиваемый резонатор}
	\end{center}

\vspace{4cm}

	\begin{center}
		\begin{minipage}{0.45\textwidth}
			\begin{center}
Выполнил студент:\\~\\~\\
Цицилин И. А.\\
группа 425
			\end{center}
		\end{minipage}
		\begin{minipage}{0.45\textwidth}
			\begin{center}
Научный руководитель:\\~\\~\\
Беседин И.С.\\
??
			\end{center}
		\end{minipage}
	\end{center}

\vspace{\fill}


	\begin{center}
г. Москва, 2018
	\end{center}

\end{titlepage}

\newgeometry{right=1in, left = 1.2in, top=1in, bottom=1in}

\tableofcontents
	\newpage

\chapter{Введение}
а ваще тут надо написать что квкомп важно, про физ реализации\\
 Здесь и в дальнейшем мы будем говорить только о сверпроводящих элементах и о куче воды заточенной под методичку\\ и пока я сюда пишу то, что не знаю куда писать
 Хочется отметить, что практически все рассмотрения поведения(динамики) кубитов, ввиду того, что трансмон--слабоангармонический осциллятор, считают ( в первом приближении), что \textbf{кубит-- гармонический осциллятор}, поэтому по сути мы практически всегда рассматриваним взаимодействие гарм.осц.-- гарм.осц. К примеру такой эффект как антикроссинг (avoided crossing) можно наблюдать и в обычных классических системах \cite{Novotny2010}. Когда резонатор связывается с кубитом, то уже не существует по отдельности резонатора и кубита, существует только система кубит--резонатор. Их дисперсионные зависимости,существовавшие по отдельности, становятся одним целым (картиночки). Это явление также широко известно в экситонных поляритоннах и др. системах. По аналогии с экситонным поляритоном, где какое-то время существует экситонн, а другое поляритон, можно сказать что наше электромагнитное возбуждение(нафиг эту аналогию, лучше просто про связ.кол.эл.контуры) перетекает между резонатором и кубитом. Чем дольше возбуждение существует в этой системе, ( и тут я запутался, возбуждение гоняет только на резонансных частотах, а резонансная частота теперь то у них общая, и когда мы посылаем сигнал на частоте резонатора мы делаем возбуждение только в резонаторе, в кубите оно затухает, видимо это вносит дополнительную декогеренцию, т.к. в резонаторе начинают гулять два сигнала теперь и резонаторный чисто и кубитно-резонаторный)( ОГО ЧЕ ПРИДУМАЛ, НАДО СПРОСИТЬ У ИЛЬИ) -- тем дольше "живет" кубит, тем больше $T_1$
	\section{Эффект Джозефсона.SQUID}
	
	\section{Сфера Блоха.Однокубитные гейты}

	\section{Дисперсионное считывание}
	При рассмотрение всех эффектов в данном разделе мы считаем кубит-трансмон-слабоангармоничным оссцилятором, поэтому в первом приближении будем считать его
		\subsection{Кубит, связанный с резонатором}

		\subsection{Два кубита, связанные через один резонатором}
ссылкa\cite{Gu2017}





\chapter{Теоретические сведения} \label{chap:theory}
	\section{Резонатор}
(вставлю картинку связанного кубита-резонатора)
связь кубит-резонатор в представлении сосредоточенных электрических элементов
	\begin{equation}
	g_{qr}=\frac{C_c}{2\sqrt{C_q C_r}}\sqrt{\omega_r \omega_q}
	\label{eq:g_qr}
	\end{equation}
выражение для константы связи, полученное через квантовомеханический рассчет, вы можете найти в \cite{Koch2007}(надо бы его сюда тоже вписать и показать, что вычисление через классику и через кв.мех должны совпадать, чисто для себя проверить)\\
Резонатор-гармонический осциллятор, он используется как фильтр, он пропускает только в определенной области частот близкой к резонансу,т.е. он выполняет не только функцию пропускания сигнала, но и блокирует пропускание 
различных шумов\cite{Naturwissenschaften2013}. Также он запасает в себе энергию электромагнитного поля и тем самым помогает её транспортировать между кубитами, связанными через один резонатор \cite{Kelly2015}
(можно здесь написать про внутреннюю и внешнюю добротность , т.к. я же каппу через прогу Ильи считаю, хорошо бы объяснить что это) \\
Резонансные частоты CPW определяются по формулам:
	\begin{align}
	\label{resfr}
	f^{(0)}_r=\frac{c}{\sqrt{\varepsilon_{eff}}}\frac{2p-1}{4l}, 
	f^{(0)}_r=\frac{c}{\sqrt{\varepsilon_{eff}}}\frac{2p}{4l}
	\end{align}
,где $v_{ph}=\frac{c}{\sqrt{\varepsilon_{eff}}}$-фазовая скорость распространения волны в резонаторе. В \eqref{resfr} первая описывает резонатор c закороченным одним концом $(\lambda/4)$ и с открытым другим, вторая описывает резонатор с обоими закороченными концами или с обоими открытыми концами $(\lambda/2)$. 
	
ДОБАВИТЬ ДОБРОТНОСТЬ\\
Резонатор представляет из себя копланарный волновод(надо поменять их местами по секциям) определенной длины, он является распределенным элементом, имеет несколько резонансных частот и при частоте близкой к резонансной может быть представлен как набор следующих сосредоточенных элементов:
	\begin{equation}
	C_r=\frac{C_0 l}{2},
	L_r=\frac{8L_0 l}{\pi^2},
	R_r=\frac{Z_{CPW}}{\alpha_0 l}
	\end{equation}
где $Z_{CPW}=\sqrt{\frac{L_0}{C_0}}$-характеристический импеданс компланарного волнового резонатора, внутренние потери в резонаторе(квазичастицы, излучение и т.п.) представляются в виде $R_r$, которое имеет прямое отношение к внутренней добротности резонатора $Q^i$, за связывание с передающей линией с импедансом $Z_0$ отвечает внешняя добротность $Q^e$.

(это я описываю, чтобы потом описать затухание в линию)
 При частотах близких к резонансным мы будем всегда работать. Используя преобразование Нортона\cite{Goppl2008} мы можем переделать нашу эквивалентую схему в схему с параллельным подключением
при этом  
	\begin{align}
	C^*&=\frac{C_c}{1+\omega_0^2 C_c^2 Z_0^2} \approx C_c
	Z^*&=\frac{1+\omega_0^2 C_c^2 Z_0^2}{\omega_0^2 C_c^2 Z_0^2} \approx \frac{1}{\omega_0^2 C_c^2 Z_0^2} 
	\end{align}
т.к. $Z_0\sim$, то$\omega_0 C_c Z_0 \ll 1$
	\subsection{Эффект Парселла}
про парселла\cite{Koch2007}	
	\subsection{Связывание кубита с резонатором}
про клешню\cite{Sank2014}\\
Связывание кубита с резонатором -- ёмкостное,	

\newpage
	\section{Копланарный волновод}%тут вроде всё
Копланарный волновод представляет из себя тонкую полосу металла -- 'сигнал' -- шириной W, отделенную от земли щелью G и длинной l.
	\begin{figure}[h]
		\begin{center}
\includegraphics[width=0.5\textwidth]{plots/cpw}
\caption{Копланарный волновод}
		\end{center}
	\end{figure}
Если d $\ll$ h,где h- высота подложки, т.е. если слой сверхпроводника очень тонкий,можно приблизительно считать,что\cite{Goppl2008}
	
	%&-значек, по которому происходит выравнивание
	\begin{align*}
	C'&=4\varepsilon_0\varepsilon_{eff}\frac{K(k_0)}{K(k'_0)},
	L'=\frac{\mu_0}{4}\frac{K(k'_0)}{K(k_0)}\\
	Z_0&=\sqrt{\frac{L'}{C'}}	=\frac{\mu_o c}{4\sqrt{\varepsilon_{eff}}}\frac{K(k'_0)}{K(k_0)}
	\end{align*}
где $k_0=\frac{W}{W+2G},k'_0=\sqrt{1-k_0^2}$, K-полный нормальный эллиптический интеграл Лежандра 1-ого рода, который считается численно.
Т.е. импеданс передающей линии зависит от двух параметров от эффективной диэлектрической проницаемости $\varepsilon_{eff}=\frac{\varepsilon_{\text{среда сверху}}+\varepsilon_{\text{среда снизу}}}{2}$, где $\varepsilon_{\text{среда сверху}}=\varepsilon_{\text{вакуума}}=1$,а в качестве подложки используют кремний или сапфир $\varepsilon_{\text{кремний}}\approx11.45$(в гопл написано 11.6 и что интересно у них эпсилон эффективная не 6.225, а 5.22 с погрешностью в 3процента???).Посчитать импеданс по этим формулам, используя удобный интерфейс вы можете на этом сайте(ссылка на сайт ильи), где имеется описание. В сверхпроводнике помимо геометрической(магнитной) индуктивности существует кинетическая индуетивность\cite{Goppl2008}, которая при некоторых условия может быть порядка геометрической. Рассмотрим характерное значение $L^k$ для Al плёнок,которые используются в наших схемах.(ссылки в гоппл надо посмотреть когда интернет будет) Но значение кин.индуктивности на 2 порядка меньше чем геом.индуктивности, поэтому ею можно пренебречь.
(ЗАБОТАЙ КИН.ИНДУКТИВНОСТЬ СОЗДАЙТЕ ЕЁ УЖЕ НАКОНЕЦ)
\newpage
	\section{Линия передачи}%выведу здесь для S11 зависимость
	




\chapter{Экспериментальная часть} \label{chap:exp}
Для разработки дизайна надо понять, что именно и как именно мы хотим исследовать на нём. Наша задача-
связать два кубита. Во многих работах связывание происходит через резонатор\cite{Kelly2015}(и еще добавлю). 
В нашей разработке хотелось бы реализовать в дальнейшем двухкубитный гейт -- CNOT, для реализации универсального квантогово компьютера. Чтобы варьировать связь между кубитами можно использовать нелинейный резонатор -- ещё один вспомогательный кубит(ссылка на стр в теории где показываю, что кубит нлр), меняя его частоту при помощи Z-гейта(ссылка на стр в теории или дальше, где я могу поговорить о том, как z гейт делать вообще), мы можешь варьировать связь, т.к. связь кубит--резонатор имеет вид \eqref{eq:g_qr},т.е. $g_{qr} \sim \sqrt{\omega_r}$, а значит использовать кубит как нелинейный связывающий резонатор целесообразно.Также исследование трехкубитной схемы необходимо для дальнейшей масштабируемости цепочки кубитов.
т.е. мы имеем первый набросок дизайна
(вставка графика)
(где-то я должен в теориии рассказать что такое иксмон и зачем нам его удобно использовать)
	
	\section{Разработка дизайна образца}
	Определим теперь какие нам необходимы допольниительные элементы для
	\begin{enumerate}
	\item считывания состояний кубитов
	\item возбуждения кубитов
	\item взаимодествия кубит--кубит
	\item взаимодействия кубит-- резонатор
	\end{enumerate}
	
		\subsection{Считывание}
Для считывания будем измерять отражение сигнала в линии передачи (\textit{transmission line}) -- $ S_{11}$
надо сюда вставить как зависит от расстояния до конца линии напряжение на ней или что вообще 
можно сказать про считывание)\cite{GlebMaster}(можно вставить умную зависимость для S11!)
Надо уметь считывать состояния кубитов по одному, для этого используются резонаторы на разных частотах \ref{chap:theory}(ссылка говно!) для каждого кубита,т.к. мы используем дисперсионное считывание(ссылочка на теорию где я говорю, что это такое), то наш резонатор отстроен по частоте от кубита, в результате, мы получаем, что сигнал, приходящийся на кубит сильно ослаблен, если его возбуждать через резонатор, и надо подавать больше мощности, если резонатор высокодобротный, то сигнал может быть слишком слабый.
	\begin{figure}[h]
		\begin{center}
\includegraphics[width=0.5\textwidth]{plots/FR}
\caption{АЧХ кубита и резонатора}
		\end{center}
	\end{figure}
Так же, если у нас есть два кубита на близкой частоте, то подавая сильный сигнал через линию передачи для возбуждения одного кубита, мы можем внести дополнительные нежелательные возбуждения в другой кубит.
		
		\subsection{Возбуждение}
Чтобы делать гейты(возбуждать кубиты) надо подавать сигнал на кубит на его резонансной частоте. Есть два способа: подавать сигнал в линию передачи,но,чтобы сигнал прошел на кубит,ему надо пройти через резонатор, или 
можно сделать отдельную линию передачи -- 'драйв', которая будет сильнее всего связана с только одним кубитом. Мы используем второй вариант -- драйв.Драйвы должны быть разнесены таким образом, чтобы можно было возбуждать один кубит разумной мощностью и при этом не возбуждать соседние кубиты( минимальный crosstalk). Наличие драйва -- наличие дополнительного источника релаксации состояния кубита, т.е. он должен иметь малую ёмкостную связь с электродом кубита.
(тут или не тут вставлять картинки из лэйаут и говорить как я там все моделировал и считал или пока вода?)
 		
 		\subsection{Взаимодействие кубит--центральный кубит}
Связь кубит--кубит -- $g_{qq}$ -- определяет время, за которое может переходить возбуждение с одного кубита на другой, т.е. определяет время двухкубитного гейта\cite{Rigetti2010} $t_{gate}\sim \frac{1}{g_{qq}}$. Использовать значение этой связи в данной работе мы будем как связь кубит--центральный кубит(наш нелинейный резонатор) 
		
		\subsection{Взаимодействие кубит--резонатор}
Чтобы считывать состояние кубита через резонатор, он должен быть связан с константой связи порядка 20--30 Мгц,
меньшее значение даёт плохой сигнал( а как это объяснить?)

	
	
	\section{Реализация произвольного однокубитного гейта}
	
\newpage


	\section{Экспериментальная установка}
		\subsection{Криостат}
		\subsection{СВЧ-оборудование}
не буду про фабрикацию ниче говорить, нафиг
\newpage
	\section{Заказанные параметры двухкубитной схемы}
\noindent Считаем $\epsilon_{eff}=6.225$\\
\noindent 3 одинаковых перестраиваемых кубита\\
\noindent Из fastcup получаем слудющие значения для ёмкостей:
	\begin{center}
		\begin{tabular}{ | c | c | c | }
   		 	\hline
   			$C_Q,fF$ & $ C_{Q-R},fF$ &$ C_{Q-Q},fF$ \\ \hline
   		 	101 & 1.94 & 1.76 \\ \hline
		\end{tabular}
	\end{center}
\noindent Перестройка по частоте от $f_{01}^{min}$=4ГГц до $f_{01}^{max}$=6.5ГГц\\
$E_{C}=\frac{e^2}{2C_Q}$ = 191 МГц\\
$E_{J_{\Sigma}}^{min} = \frac{(hf_{01}^{min}+E_{C})^2}{8E_{C}}=$11.5ГГц \\
$E_{J_{\Sigma}}^{max} = \frac{(hf_{01}^{max}+E_{C})^2}{8E_{C}}=$29.3ГГц \\
$E_{J}=\frac{\Phi_0 I_c}{2\pi}\Rightarrow$ \\
$I_{c_{\Sigma}}^{min}=$23.1нА $I_{c_{\Sigma}}^{max}=$59нА$\Rightarrow$
получаем крит.токи через джозефсоны(полусумма и полуразность максимального и минимального)\\
$I_{c_{1}}=$41нА $I_{c_{2}}=$18нА\\
при площадях $S_1 = 0.235*0.117 \text{мкм}^2$  $S_2 = 0.156*0.078\text{мкм}^2$  \\
$j_2=\frac{I_{c_{2}}}{S_2}=1.49\frac{\mu A}{{\mu m}^2}$
	\begin{center}
		\begin{tabular}{ | c | c | c |}
   		 	\hline
   			  &1 &2 \\ \hline
    			 $I_c,nA$ & 41 & 18 \\ \hline
    	 		$S,{\text{мкм}}^2$ & 0.235*0.117 & 0.156*0.078  \\ \hline
    	 		 $j,\frac{\mu A}{{\mu m}^2}$  &\multicolumn{2}{|c|}{1.49}\\ \hline
		\end{tabular}
	\end{center}
	
Получились частоты перестройки с такими параметрами 
от $f_{01}^{min}$=4.17 ГГц у боковых и 4.13 ГГц у центрального до $f_{01}^{max}$=6.65ГГц у боковых и 6.6ГГц у центрального.

Диапазон констант связи кубит--центральный кубит в зависимости от частоты кубитов:
	\begin{center}
		\begin{tabular}{ | c | c | c |}
  	  	\hline
   		  &для $f_{01}^{min}$ &для $f_{01}^{max}$ \\ \hline
    		  $g_{Q-Q}$,МГц &35,2  & 56,2 \\ \hline
		\end{tabular}
	\end{center} 
	
Диапазон констант связи кубит--резонатор в зависимости от частоты кубита:
	\begin{center}
		\begin{tabular}{ | c | c | c |}
  	  	\hline
   		  f резонатора,ГГЦ&для $f_{01}^{min}$ &для $f_{01}^{max}$ \\ \hline
    		  6.85 &23,1 МГц & 29,2 МГц \\ \hline
    		  7.15 &24,1 МГц & 30,5 МГц \\ \hline
   	 	  7 &23,6 МГц & 29,9 МГц \\ \hline
		\end{tabular}
	\end{center} 
	
	
	\bibliographystyle{ugost2008ls}
	\bibliography{Thesis_Ivan.bib}
\end{document}