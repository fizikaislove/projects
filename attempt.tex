\documentclass[12pt,a4paper,twocolumn]{article}
\usepackage[utf8]{inputenc}
\usepackage[russian]{babel}
\usepackage[OT1]{fontenc}
\usepackage{amsmath}
\usepackage{amsfonts}
\usepackage{amssymb}
\usepackage{graphicx}
\usepackage{wrapfig}
\usepackage{multicol}
\usepackage[left=2cm,right=2cm,top=2cm,bottom=2cm]{geometry}


\newcommand{\com}[2]{\ensuremath{ [{#1} , {#2}]}}
\newcommand{\sx}[1]{\ensuremath{{\hat{\sigma}_{x_{#1}}}}}
\newcommand{\sy}[1]{	\ensuremath{{\hat{\sigma}_{y_{#1}}}}}
\newcommand{\sz}[1]{	\ensuremath{{\hat{\sigma}_{z_{#1}}}}}
\newcommand{\spm}[1]{\ensuremath{{\hat{\sigma}_{\pm_{#1}}}}}
\newcommand{\sm}[1]{	\ensuremath{{\hat{\sigma}_{-_{#1}}}}}
\newcommand{\spp}[1]{\ensuremath{{\hat{\sigma}_{+_{#1}}}}}
\newcommand{\had}{\ensuremath{\hat{a}^+}}
\newcommand{\op}[1]{\ensuremath{\hat{{#1}}}}
\newcommand{\ha}{\ensuremath{\hat{a}}}



\author{Цицилин Иван}

\begin{document}
используемые коммутаторы:
\begin{align}
	\spm{i}=\frac{\sx{i}\pm i\sy{i}}{2} \\
	\com{\spm{1}}{\sx{1}}=\pm\sz{1} \\ 
	\com{\spp{i}}{\sm{i}} = \sz{i} \\
	\com{\spp{i}}{\sz{i}} = -2\spp{i} \\
	\com{\sm{i}}{\sz{i}} = 2\sm{i} \\ 	
 	\com{\ha}{\had}=1 \\
 	\com{\sz{i}}{\sx{i}}=2i\sy{i}
\end{align}
При этом, еще коммутатор любой сигма-матрицы и оператора рожд.уничт. равен 0\\
\\
Гамильтониан кубита:\\
$\op{H}_1 = \frac{\hbar\omega_{q_1}}{2}\sz{1}+\hbar\omega_{r} \had\ha+ \hbar g_1(\spp{1}\ha+\sm{1}\had)$\\
В дисперсионном режиме выполняется $\frac{g}{\Delta}\lll 1$,где$\Delta = \omega_q - \omega_r$- отсройка частоты кубита от частоты резонатора.
Сделаем унитарное преобразование $\op{U}=e^{\frac{g_1}{\Delta}(\spp{1}\ha-\sm{1}\had)}$
для этого,воспользуемся леммой хаусдорфа:\\
$e^{\op{A}} \op{B} e^{-\op{A}} = \op{B}+\com{\op{A}}{\op{B}}+\frac{1}{2}\com{\op{A}}{\com{\op{A}}{\op{B}}}+...$
в первом приближении по параметру малости $\frac{g_1}{\Delta}$ можно ограничиться только первым коммутатором
получаем, что $\op{H}_{trans}$ = $\op{H}_1+\com{\frac{g_1}{\Delta}(\spp{1}\ha+\sm{1}\had)}{\op{H}_1}$
наши вычисления разбиваются на 3 этапа, при этом, держим в уме $\frac{g_1}{\Delta}$ 
\begin{align}
\com{\spp{1}\ha-\sm{1}\had}{\frac{\hbar\omega_{q_1}}{2}\sz{1}} \\
\com{\spp{1}\ha-\sm{1}\had}{\hbar\omega_{r} \had\ha}\\
\com{\spp{1}\ha-\sm{1}\had}{\hbar g_1(\spp{1}\ha+\sm{1}\had)}
\end{align}
1) посчитаем выражение (8): (8) = \\
$\com{\spp{1}}{\sz{1}}\frac{\hbar\omega_{q_1}}{2}\ha-\com{\sm{1}}{\sz{1}}\frac{\hbar\omega_{q_1}}{2}\had \stackrel{(4)(5)}{=} \\
-\hbar\omega_{q_1} \spp{1}\ha-\hbar\omega_{q_1}\sm{1}\had =-\hbar\omega_{q_1}(\sm{1}\had+\spp{1}\ha)  \\
 $
2) посчитаем выражение (9): (9) = \\
$\hbar\omega_{r}*\\
(\spp{1}\ha\had\ha-\had\ha\spp{1}\ha-
\sm{1}\had\had\ha+\had\ha\sm{1}\had)
\\ \stackrel{(6)}{=}\hbar\omega_{r}
(\spp{1}(\had\ha+1)\ha-\had\ha\ha\spp{1}-\\
\sm{1}\had(\ha\had-1)+\had\ha\had\sm{1})=\\
\hbar\omega_{r}(\spp{1}\ha+\sm{1}\had)
$ в результате (9) и (8) получаем:
$
\hbar(\omega_r-\omega_{q_1})(\spp{1}\ha+\sm{1}\had)=-\hbar\Delta(\spp{1}\ha+\sm{1}\had)
$
т.к. это слагаемое умножается на $\frac{g_1}{\Delta}$, то получаем:
$
-\hbar g_1(\spp{1}\ha+\sm{1}\had)
$, которое уничтожает слагаемое в исходном гамильтониане, отвечающее взаимодействию кубит-резонатор\\
3) посчитаем выражение (10): (10)=\\
$
\com{\spp{1}\ha-\sm{1}\had}{\hbar g_1(\spp{1}\ha+\sm{1}\had)}\stackrel{(*)}{=}
$ \\т.к. $\com{\op{A}}{\op{A}}=0$,то 

$\stackrel{(*)}{=} \hbar g_1(\com{\spp{1}\ha}{\sm{1}\had}-\com{\sm{1}\had}{\spp{1}\ha})=\\
=2\hbar g_1\com{\spp{1}\ha}{\sm{1}\had}=2\hbar g_1(\spp{1}\ha\sm{1}\had-\sm{1}\had\spp{1}\ha)\stackrel{(6)}{=}
2\hbar g_1(\spp{1}\sm{1}(\had\ha+1)-\sm{1}\had\spp{1}\ha)=2\hbar 
g_1(\com{\spp{1}}{\sm{1}}\had\ha+\spp{1}\sm{1})\stackrel{(3)}{=}
2\hbar g_1(\sz{1}\had\ha+\spp{1}\sm{1})
$Отсюда получаем, что наш гамильтониан в дисперсионном режиме имеет представление:
$
\op{H}_{trans}=\frac{\hbar\omega_{q_1}}{2}\sz{1}+\hbar\omega_{r} \had\ha+ \frac{2\hbar g_1^2}{\Delta_1}(\sz{1}\had\ha+\spp{1}\sm{1})
$,т.к. $\spp{1}\sm{1}=\frac{1}{2}(\op{E}-\sz{1})$\\Получили что произойдет с гамильтонианом одного кубита вида$\op{H}_1$ при данном унитарном преобразовании
\begin{align}
\op{H}_{trans}\approx\hbar(\omega_r+\frac{2g_1^2}{\Delta_1}\sz{1})\had\ha+
					  \frac{\hbar}{2}(\omega_{q_1}-\frac{2g_1^2}{\Delta_1})\sz{1} 
\end{align}
Для нашей двухкубитной схемы( где есть два кубита, связанные как-то через общий резонатор), гам.системы имеет вид: \\$\op{H_{sys}}=\frac{\hbar\omega_{q_1}}{2}\sz{1} +\frac{\hbar\omega_{q_2}}{2}\sz{2}+\hbar\omega_{r} \had\ha+
\hbar g_1(\spp{1}\ha+\sm{1}\had)+ \hbar g_2(\spp{2}\ha+\sm{2}\had)$ \\сделаем теперь преобразование $\op{U}=\op{U_1}\op{U_2}$, где $\op{U_i}=
e^{\frac{g_i}{\Delta_i}(\spp{i}\ha-\sm{i}\had)}
$ После первого преобразования гам.1ого кубита станет равным (11), посмотрим, как оно повлиет на гамильтониан 2кубита,  так же по лемме Хаусдорфа:
$\op{H}_{2_{trans}}=\op{H}_2+\com{\frac{g_1}{\Delta_1}(\spp{1}\ha-\sm{1}\had)}{\op{H_2}}$, т.к 
$\com{\sigma_i^1}{\sigma_j^2}=0  \forall i\in Q1,j\in Q2$, т.к. это две разные системы,то ненулевым является слагаемое вида (10):\\
$
\frac{\hbar g_1 g_2}{\Delta_1}\com{\spp{1}\ha-\sm{1}\had}{\spp{2}\ha+\sm{2}\had}=\\
\frac{\hbar g_1 g_2}{\Delta_1}(\spp{1}\sm{2}\com{\ha}{\had}-\sm{1}\spp{2}\com{\had}{\ha})=\\
\frac{\hbar g_1 g_2}{\Delta_1}(\spp{1}\sm{2}+\sm{1}\spp{2})\com{\ha}{\had}\stackrel{(6)}{=}\\ \frac{\hbar g_1 g_2}{\Delta_1}(\spp{1}\sm{2}+\sm{1}\spp{2})
$
т.е. после первого преобразования, наш гам. системы имеет вид:\\
$
\op{H}_{sys_{trans}}\approx\hbar(\omega_r+\frac{2g_1^2}{\Delta_1}\sz{1})\had\ha+
					  \frac{\hbar}{2}(\omega_{q_1}-\frac{2g_1^2}{\Delta_1})\sz{1}+
				      \frac{\hbar\omega_{q_2}}{2}\sz{2}+					  
					  \frac{\hbar g_1 g_2}{\Delta_1}(\spp{1}\sm{2}+\sm{1}\spp{2})
					  $
\\					 
			МОЙ НАСТОЯЩИЙ ГАМИЛЬТОНИАН		 $\frac{\hbar\omega_{Q_1}}{2}\sz{1}+\frac{\hbar \omega_{Q_2}}{2}\sz{2}+\hbar g_2y \sy{2}(\ha - \had)+ \hbar g_2x \sx{2}(\ha + \had)+\hbar g_1 \sx{1}(\ha + \had)+\hbar \omega_r \had\hat+ \hbar g_{12} \sx{2}\sx{1}$













\newpage
Если рассматривать эквивалентную электрическую цепь ,то кинетическая энергия имеет вид:
$T=\frac{1}{2det(C)}*\\
(C_1 (C_2 + C_{Q_2}) + C_{Q_2} C_r + C_2 (C_{Q_2} + C_r)) q^2_1 +\\
+ 
 2 C_1 (C_2 + C_{Q_2}) q_1 q_2 +\\
 + (C_1 + C_{Q_1}) (C2 + C_{Q_2}) q^2_2 +\\
 + 2 C_1 C_2 q_1 q_3 +\\
 +  2 C_2 (C_1 + C_{Q_1}) q_2 q_3 +\\
 + (C_{Q_1} (C_2 + C_r) + C_1 (C_2 + C_{Q_1} + C_r)) q^2_3$\\
 ,где  $det(C)=C_1 C_2 C_{Q_1} + C_1 C_2 C_{Q_2} + C_1 C_{Q_1} C_{Q_2} + C_2 C_{Q_1} C_{Q_2} + C_1 C_2 C_r + 
 C_2 C_{Q_1} C_r + C_1 C_{Q_2} C_r + C_{Q_1} C_{Q_2} C_r$
 члены имеющие вид $q_1q_3$ отвечают за связь кубит-кубит,а $q_1q_2,q_2q_3$ кубит-общий резонатор, а потенциальная имеет вид:
$\Pi=E_{J_1}\cos(\phi_1)+E_{J_2}\cos(\phi_3)-\frac{\phi_2^2}{2L_r}-\frac{(\phi_2-\phi_3)^2}{2L_2}=\\
E_{J_1}\cos(\phi_1)+E_{J_2}\cos(\phi_3)-\frac{\phi_2^2}{2}(\frac{1}{L_r}+\frac{1}{L_2})-\frac{\phi_3^2}{2L_2}+\frac{\phi_2\phi_3}{L_2}$\\
член, отвечающие за взаимодействие кубит-резонатор здесь - $\frac{\phi_2\phi_3}{L_2}$
Теперь проквантуем кубиты и резонаторы.
"Координаты "$q_1\phi_1$- для 1ого,$q_3\phi_3$- для 2ого,$q_2\phi_2$- для резонатора
Для кубитов, считая, что $C_1,C_2 \lll C_i$ любого элемента, получаем вид T в нулевом приближении:\\
$
\frac{2q_1^2}{C_{Q_1}}+\frac{4 C_1 q_1 q_2}{C_{Q_1} C_r}+\frac{2q_2^2}{C_r}+\frac{4 C_1 C_2 q_1 q_3}{C_{Q_1}
 C_{Q_2}C_r}+\frac{4 C_2 q_2 q_3}{C_{Q_2} C_r}+\frac{2q_3^2}{C_{Q_2}}
$\\
а значит, что наш гамильтониан электрической цепи \op{H} =\\
$
\frac{2q_1^2}{C_{Q_1}}+\frac{4 C_1 q_1 q_2}{C_{Q_1} C_r}+\frac{2q_2^2}{C_r}+\frac{4 C_1 C_2 q_1 q_3}{C_{Q_1}
 C_{Q_2}C_r}+\frac{4 C_2 q_2 q_3}{C_{Q_2} C_r}+\frac{2q_3^2}{C_{Q_2}}+
\frac{{\phi_1}^2}{2L_1}+\frac{{\phi_3}^2}{2}(\frac{1}{2L_2}-\frac{1}{L_2})-\frac{\phi_2^2}{2}(\frac{1}{L_r}+\frac{1}{L_2})+\frac{\phi_2\phi_3}{L_2}
$
\end{document}